%O BibTeX permite que você utilize um arquivo externo para organizar suas referências bibliográficas (.bib). A maneira de citar a referência no documento TeX não é alterada, porém basta informar dentro do documento TeX o nome do arquivo com as referências.
%Vamos supor que o arquivo contendo as referências seja "ref.bib" e o documento TeX seja "teste.tex". Para vincular o documento .bib ao documento TeX utilizamos o comando \bibliography, conforme o exemplo abaixo
\documentclass[11PT,preprint,authoryear]{elsarticle}
\includeonly{Conclus,Section}
%\documentclass[11pt]{article}
%\usepackage{cite}
%\usepackage{natbib}
\begin{document}
\title{CENSORED REGRESSION MODEL}
\author{Prof. Aldo M. Garay}
\maketitle
\section{Introduction}

Regression models with normal observational errors  are usually
applied  to model symmetrical data. However, it is well-known that
several phenomena are not always in agreement with the assumptions
of the normal model, yielding data  with heavier tails or skewed or
multimodal distributions. A good alternative is to consider errors
with a more flexible class of distributions, such as, the Student-t
distribution. For instance, \cite{fernandez1999multivariate} discuss
some inferential procedures in regression models with Student-t
distribution for the errors. \cite{Pinheiro01} proposed a
multivariate Student-t linear mixed model (t-LME) and demonstrated its
robustness against outliers through extensive simulations.
\cite{lin2006bayesian} and \cite{lin2007bayesian} developed some
additional tools for t-LME from likelihood-based and Bayesian
perspectives. More recently, \cite{IbacachePulgar20111462}, proposed
local influence measures in the Student-t partially linear
regression model. Other existing methods for robust estimation are
based on the class of scale mixtures of normal  (SMN) distributions. This class of
distributions is symmetric and thick-tailed and includes as special
cases many symmetric distributions, such as, the normal, Pears
type VII, Student-t, slash and the contaminated normal. For an
overview and applications, see for instance, \cite{Andrews1974},
\citep{Lange93} and \cite{Rosa2003}.

In this work, we are interested in fitting  regression models when
the responses are possibly   censored.  Censoring occurs in several
practical situations, for reasons such as
limitations of measuring equipment or from experimental
design. Roughly speaking, a censored observation  contains only
partial information about an  event of interest.  

Another referencies is \cite{Branco2012}

%\section*{References}
\bibliographystyle{elsarticle-harv}
\bibliography{teste}

\end{document}

